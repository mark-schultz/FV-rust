\documentclass{beamer}
\usetheme{ucsandiego} 

\usepackage{color} 
\usepackage{graphicx} % external images
\usepackage[font=small]{caption} % caption numbers
\usepackage{tikz}
\usepackage{xcolor}

\usepackage{mathtools, amssymb, physics}

\theoremstyle{definition}




\title{Implementing the FV FHE Scheme}
\author{Mark Schultz} 
\institute{University of California San Diego}
\date{10 May 2022} % custom date

% Comment this section out to disable table of contents at section begin.
\AtBeginSection[]
{
\begin{frame}
\frametitle{Contents}
% This will display the table of contents and highlight the current section.
\tableofcontents[currentsection] 
\end{frame}
}

\newcommand{\enc}{\mathsf{Enc}}
\newcommand{\dec}{\mathsf{Dec}}
\newcommand{\Nset}{\mathbb{N}}


\begin{document}

{
\usebackgroundtemplate{
    \tikz[overlay,remember picture]
    \node[opacity=0.05, at=(current page.south east),anchor=south east,inner sep=0pt] {
        \includegraphics[height=\paperheight,width=\paperwidth]{imgs/trident-30-4x3-right-bot}};
}
\begin{frame} 
\titlepage
\end{frame}
}

\section{Introduction}

\begin{frame}
	\frametitle{Who am I?}
	\begin{itemize}
		\item 4th year PhD Student at U.C.S.D\pause
\begin{itemize}
	\item Working with Daniele Micciancio in Lattice-based Cryptography\pause{}
\end{itemize}
\item Want to get industry experience this summer\pause{}
\begin{itemize}
	\item Especially on implementations
\end{itemize}
	\end{itemize}
\end{frame}

\begin{frame}
	\frametitle{The Challenge}
	\begin{itemize}
		\item Implement a Fully Homomorphic Encryption Scheme!\pause{}
		\begin{itemize}
			\item The Fan-Vercauteren Scheme (FV) in particular\pause{}
		\end{itemize}
	\item Implemented everything but Bootstrapping
	\end{itemize}
\end{frame}

\section{Theoretical Description}

\begin{frame}
	\frametitle{What is FHE?}
	\begin{itemize}
		\item PKE that supports \emph{privacy homomorphism}.\pause{}
		\begin{itemize}
			\item Method to compute
			\begin{equation*}
			(f, \enc_{pk}(m_0),\enc_{pk}(m_1))\mapsto \enc_{pk}(f(m_0,m_1))	
			\end{equation*}\pause{}
			\item $f \in \{+, \times\}$ is enough\pause{}.
		\end{itemize}
		\item Many early cryptosystems are \emph{Partially Homomorphic}\pause{}
		\item First FHE: Gentry 2009\pause{}
		\begin{itemize}
			\item Combines \emph{Somewhat Homomorphic Encryption} and Bootstrapping.\pause{}
		\end{itemize}
	\item Also \emph{Leveled FHE}.\pause{}
	\begin{itemize}
		\item FV is Leveled FHE.
	\end{itemize}
	\end{itemize}
\end{frame}

\begin{frame}
	\frametitle{What is FV?}
	\begin{itemize}
		\item RLWE-based version of Brakerski's ``Scale-Invariant'' FHE scheme\pause{}
		\item Alternatively, variant of the LPR Cryptosystem that uses relinearization-based multiplication\pause{}
		\begin{itemize}
			\item I will explain things from this perspective.
		\end{itemize}
	\end{itemize}
\end{frame}

\begin{frame}
	\frametitle{Ring LWE}
	\begin{itemize}
		\item Throughout, fix $R_q = \mathbb{Z}_q[x] / (2^d+1)$ for $d = 2^n$, and $q\in\Nset$.
		
		\graybox{Ring LWE Assumption}{
		For random $s\in R_q$, and a distribution $\chi$ supported on $R_q$, the RLWE assumption is that for $a(x)\gets R_q, u(x)\gets R_q$, $e(x)\gets \chi$:
		\begin{equation*}
			(a(x), a(x)s(x) + e(x))\approx_c (a(x), u(x)).
		\end{equation*}}\pause{}
	\item $\chi$ a discrete Gaussian of parameter $\sigma$.
	\end{itemize}
	
\end{frame}

\begin{frame}
	\frametitle{LPR Cryptosystem}
	
	\begin{itemize}
		\item KeyGen: $sk\gets \chi$, and $pk = (a(x), a(x)s(x) + b(x))$ is an RLWE sample\pause{}
		\item Encryption: Sample $u(x)\gets R_q$, $e_0(x),e_1(x)\gets \chi$, and outputs
		\begin{equation*}
			u(x)\begin{pmatrix}
				a(x)s(x)+e(x)\\
				-a(x)
			\end{pmatrix} +\begin{pmatrix}
			e_0(x)\\e_1(x)
		\end{pmatrix}  + \begin{pmatrix}
		\Delta m(x)\\ 0
	\end{pmatrix}.
		\end{equation*}\pause{}
	\begin{itemize}
		\item Easily supports additive homomorphism\pause{}
	\end{itemize}
	\item Decryption: For a ciphertext $\begin{pmatrix}
		c_0(x)\\c_1(x)
	\end{pmatrix}$, compute $c_0(x) + s(x)c_1(x)$, and perform simple error-correction procedure.
	\end{itemize}
\end{frame}

\begin{frame}
	\frametitle{Relinearization}
	\begin{itemize}
		\item Idea: view decryption $\dec_{s}(c_0, c_1) = c_0 + c_1s = f_{c_0,c_1}(s)$ as a degree-1 polynomial in $s$, where
		\begin{equation*}
			c_0+c_1s = \Delta m + e(x).
		\end{equation*}\pause
	\item For two ciphertexts, one can multiply these polynomials
\begin{equation*}
	\dec_s(c_0,c_1)\dec_s(c_0',c_1') = \Delta^2 mm' + \Delta\times E
\end{equation*}	\pause{}
\item After scaling down by $\Delta$, one obtains an encryption of $mm'$
	\end{itemize}
\end{frame}

\begin{frame}
	\frametitle{Relinearization Difficulties}
	\begin{enumerate}
		\item The polynomial is now of degree \alert{two} in $s$\pause
		\begin{itemize}
			\item additionally, $E$ may be large.\pause{}
		\end{itemize}
		\item Relinearization is a technique to address both of these issues\pause
		\begin{itemize}
			\item I will focus on discussing the first issue
		\end{itemize}
	\end{enumerate}
\end{frame}

\begin{frame}
	\frametitle{Reducing Degrees}
	\begin{itemize}
		\item The goal is to \alert{linearize} the degree 2 polynomial, i.e. find $c_0', c_1'$ such that
		\begin{equation*}
			c_0 + c_1s + c_2s^2 = c_0' + c_1's
		\end{equation*}\pause{}
	\item Done by utilizing certain encrypted forms of $s^2$\pause{}
	\begin{itemize}
		\item Called the \emph{relinearization key} $rk$\pause{}
\begin{itemize}
	\item In particular, it satisfies $rk_0 + rk_1s = s^2+e$\pause{}
	\item Can multiply by $c_2$ and subtract\pause{}
	\item Issue: $c_2 e$ may be \alert{large}
\end{itemize}
	\end{itemize}
	\end{itemize}
\end{frame}

\begin{frame}
	\frametitle{Making $c_2$ smaller}
	\begin{itemize}
	\item Idea is to take a base $T$ decomposition $c_2 = \sum_i c_{2,i}T^i$\pause{}
	\item Each $c_{2,i}$ \alert{small} now\pause{}
	\begin{itemize}
		\item Downside: Relinearization key needs encryptions of $T^is^2$ for each $i$
	\end{itemize}
\end{itemize}
\end{frame}


\section{Implementation Description} 

\begin{frame}
	\frametitle{High Level Implementation Overview}
	\begin{itemize}
		\item Language: Rust\pause{}
		\item Used crates for Big-Int arithmetic + a PRG\pause{}
		\item Focused implementation on straightforward algorithms\pause{}
		\item All of FV implemented except Bootstrapping\pause{}
		\begin{itemize}
			\item Only implemented one of the two relinearization techniques.\pause{}
		\end{itemize}
	\end{itemize}
\end{frame}

\begin{frame}
	\frametitle{The Straightforward Algorithms}
	\begin{itemize}
		\item Polynomial Arithmetic\pause{}
		\begin{itemize}
			\item Power of Two Cyclotomics\pause{}
			\item Schoolbook Multiplication: $O(n^2)$\pause{}
			\begin{itemize}
				\item Karatsuba $O(n^{1.6})$, Toom-Cook $O(n^{1+\epsilon})$\pause{}
				\item Number-Theoretic Transform: $O(n\log_2n)$.
			\end{itemize}
		\end{itemize}
		\item Gaussian Sampling\pause{}
		\begin{itemize}
			\item Inverse-CDT with f64's.\pause{}
			\begin{itemize}
				\item Turns $[0,1]$ samples to samples of \emph{any} distribution\pause{}
			\end{itemize}
			\item Precision a little low.\pause{}
			\item Big hit to make side-channel resistant\pause
			\item Space complexity that scales poorly with $\sigma$.
		\end{itemize}
		\item Relinearization via Digit Decomposition
	\end{itemize}
\end{frame}

\begin{frame}
	\frametitle{Parts of Implementation Working}
	\begin{itemize}
		\item Everything but Bootstrapping!\pause{}
		\begin{itemize}
			\item (And potentially extremely large $\sigma$ Gaussian sampling)\pause{}
		\end{itemize}
		\item Concretely
		\begin{itemize}
			\item All PKE algorithms supported\pause{}
			\item Addition and Multiplication work\pause{}
			\item Relinearization works
		\end{itemize}
	\end{itemize}
\end{frame}

\begin{frame}
	\frametitle{Some Timings}
	\begin{itemize}
		\item Due to Rust implementation, \emph{extremely} efficient.\pause{}
		\item Opt-level = 3 (and full-size parameters) leads to multiplications in $\approx 5$ seconds\pause{}
		\item Funny story: Initial timing \emph{much} worse\pause{}
		\begin{itemize}
			\item $\approx 180$s \pause{}
			\item Due to choice of base $T = 2$ for relinearization\pause{}
			\begin{itemize}
				\item Implementation was implicitly using \emph{big int} arithmetic for \emph{binary} computations\pause{}
			\end{itemize}
		\end{itemize}
	\end{itemize}
\end{frame}


\section{Practical Demonstration} 

\section{Conclusion}

\begin{frame}
	\frametitle{Conclusion}
	\begin{itemize}
		\item Plenty that could still be done\pause{}
\begin{itemize}
	\item Bootstrapping\pause{}
	\item Faster polynomial arithmetic\pause{}
	\item Better Gaussian sampling\pause{}
\end{itemize}
\item Fun Implementation\pause
\begin{itemize}
	\item The easy parts were hard, and the hard parts were easy
\end{itemize}
	\end{itemize}
\end{frame}




\end{document}

